\documentclass[12pt]{article}

\usepackage{answers}
\usepackage{setspace}
\usepackage{graphicx}
\usepackage{enumitem}
\usepackage{multicol}
\usepackage{mathrsfs}
\usepackage[margin=1in]{geometry} 
\usepackage{amsmath,amsthm,amssymb}
 
\newcommand{\N}{\mathbb{N}}
\newcommand{\Z}{\mathbb{Z}}
\newcommand{\C}{\mathbb{C}}
\newcommand{\R}{\mathbb{R}}

\DeclareMathOperator{\sech}{sech}
\DeclareMathOperator{\csch}{csch}
 
\newenvironment{theorem}[2][Theorem]{\begin{trivlist}
\item[\hskip \labelsep {\bfseries #1}\hskip \labelsep {\bfseries #2.}]}{\end{trivlist}}
\newenvironment{definition}[2][Definition]{\begin{trivlist}
\item[\hskip \labelsep {\bfseries #1}\hskip \labelsep {\bfseries #2.}]}{\end{trivlist}}
\newenvironment{proposition}[2][Proposition]{\begin{trivlist}
\item[\hskip \labelsep {\bfseries #1}\hskip \labelsep {\bfseries #2.}]}{\end{trivlist}}
\newenvironment{lemma}[2][Lemma]{\begin{trivlist}
\item[\hskip \labelsep {\bfseries #1}\hskip \labelsep {\bfseries #2.}]}{\end{trivlist}}
\newenvironment{exercise}[2][Exercise]{\begin{trivlist}
\item[\hskip \labelsep {\bfseries #1}\hskip \labelsep {\bfseries #2.}]}{\end{trivlist}}
\newenvironment{solution}[2][Solution]{\begin{trivlist}
\item[\hskip \labelsep {\bfseries #1}]}{\end{trivlist}}
\newenvironment{problem}[2][Problem]{\begin{trivlist}
\item[\hskip \labelsep {\bfseries #1}\hskip \labelsep {\bfseries #2.}]}{\end{trivlist}}
\newenvironment{question}[2][Question]{\begin{trivlist}
\item[\hskip \labelsep {\bfseries #1}\hskip \labelsep {\bfseries #2.}]}{\end{trivlist}}
\newenvironment{corollary}[2][Corollary]{\begin{trivlist}
\item[\hskip \labelsep {\bfseries #1}\hskip \labelsep {\bfseries #2.}]}{\end{trivlist}}
 
\begin{document}
 
% --------------------------------------------------------------
%                         Start here
% --------------------------------------------------------------
 
\title{Math 40 Logarithmic/Exponential Equations Boardwork}%replace with the appropriate homework number
\author{Bryan Garcia\\ %replace with your name
Math 40 - Spring 2018} %if necessary, replace with your course title
 
\maketitle
%Below is an example of the problem environment

\begin{enumerate}[label=\alph*)]
    \item $7.2^x-65=0$
    \item $3\ln(x)=-3$
    \item $\log_x(\log_3(27))=3$
    \item $\log_3(x-4)+\log_3(x+4)=2$
    \item $\log_6(x+7)-\log_6(x-2)=\log_6(5)$
    \item $4+5e^x=9$
    \item $1000^{2x+1}= 100^{3x}$
    \item $3^{3x}\cdot3^{x^2}=81$
\end{enumerate}

\newpage
%Below is the solution environment
\begin{solution}{}
\begin{enumerate}[label=\alph*)] 
    \item \textbf{Game Plan}: isolate the exponential part so that we can bring the exponent down with a logarithm and solve. Let's go:
    \begin{align*}
        7.2^x - 65 &= 0 \\
        7.2^x &= 65 \\
        \ln(7.2^x) &= \ln(65) \\
        x\cdot\ln(7.2) &= \ln(65) \\
    \end{align*}
    $$\boxed{x = \frac{\ln(65)}{\ln(7.2)}}$$



    \item \textbf{Game Plan}: isolate the logarithm part so that we can exponentiate the logarithm and solve. Let's go:
    \begin{align*}
        3\ln(x) &= -3 \\
        \ln(x) &= -1 \\
        e^{\ln(x)} &= e^{-1} \\
        x &= e^{-1} \\
    \end{align*}
    $$\boxed{x =\frac{1}{e}}$$
    
    
    \item \textbf{Game Plan}: Simplify the log on the inside, then exponentiate and solve the equation:
    
    \begin{align*}
        \log_x(\log_3(27)) &= 3\\
        \log_x(3) &=3 \\
        x^{\log_x(3)} &= x^3 \\
        3 &= x^3 \\
        \sqrt[3]{3} &= \sqrt[3]{x^3}
    \end{align*}
    $$\boxed{x = \sqrt[3]{3}}$$
    $$ $$
    $$ $$
    
    \item \textbf{Game Plan}: Use our properties of logarithms to make a SINGLE log. Once we do that, we can exponentiate to get rid of the log, and then solve for x.
    \begin{align*}
        \log_3(x-4) + \log_3(x+4) &= 2 \\
        \log_3{(x-4)(x+4)} &= 2 \\
        \log_3{(x^2-16)} &= 2 \\
        3^{\log_3{(x^2-16)}} &= 3^2 \\
        x^2-16  &= 9 \\
        x^2 &= 25 \\
        \sqrt{x^2} &= \sqrt{9} \\
    \end{align*}
    $$\boxed{x = \pm 3}$$
    
    \item \textbf{Game Plan}: Again, use our properties of logarithms to make a SINGLE log. Once we do that, we can exponentiate to get rid of the log, and then solve for x.
    \begin{align*}
        4+5e^x &= 9 \\
        5e^x &= 5\\
        e^x &= 1 \\
        ln(e^x) &= ln(1) \\
        xln(e) & = 0 \\
        x\cdot 1 &= 0 \\
    \end{align*}
    $$\boxed{x = 0}$$
    
    \item \textbf{Game Plan}: Let's isolate our exponential function, then use a \textbf{natural log} to be able to solve for x.
    \begin{align*}
        \log_6(x+7) - \log_6(x-2) &= \log_6(5) \\
        \log_6{\bigg{(}\frac{x+7}{x-2}\bigg{)}} &= \log_6(5) \\
        6^{\log_6{(\frac{x+7}{x-2}})} &= 6^{\log_6(5)} \\
        \frac{x+7}{x-2} &= 5 \\
        x+7 &= 5(x-2) \\
        x+7 &= 5x - 10 \\
        17 &= 4x
    \end{align*}
    $$\boxed{x = \frac{17}{4}}$$
    
    \item \textbf{Game Plan}: Rewrite the bases of the exponential parts so that they match. Once we do that, we can set the exponents equal to each other and solve:
    \begin{align*}
        1000^{2x+1} &= 100^{3x} \\
        (10^3)^{2x+1} &= (10^2)^{3x} \\
        10^{3(2x+1)} &= 10^{2(3x)} \\
        3(2x+1) &= 2(3x) \\
        6x + 3 & =  6x \\
        3 &= 0 \\
        Contradiction. &
    \end{align*}
    $$\boxed{No\ Solution.}$$
    
    \item \textbf{Game Plan}: Let's rewrite everything with a base of 3 so that we can then set the exponents equal to each other:
    \begin{align*}
        3^{3x}\cdot3^{x^2} &= 81 \\
        3^{3x}\cdot3^{x^2} &= 3^4 \\
        3^{3x+x^2} &= 3^4 \\
        x^2 + 3x &= 4 \\
        x^2 + 3x - 4 &= 0 \\
        (x+4)(x-1) &= 0
    \end{align*}
    $$\boxed{x = -4, x = 1}$$
    
    
\end{enumerate}
\end{solution}


\end{document}
